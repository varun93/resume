\cvsection{Projects}
\begin{cventries}
    \cventry
    {Android Application}
    {Bubble Pic Frames}
    {Surathkal, India}
    {December. 2013 - April. 2014}
    {
      \begin{cvitems}
        \item {Developed an android application which enables users to create collage using bubbles. Available on  \href{https://play.google.com/store/apps/details?id=com.applications.bubblepicframes&hl=en}{\textit{\textbf{Playstore.}}}}
        \item {Implemented algorithms to move and scale the bubbles with user gestures.}
        \item {Integrated Libgdx Physics Engine to simulate the movement of a real-world bubble, Aviary Image SDK for adding special effects.}
      \end{cvitems}
    }
    \cventry
    {Researcher for Data Mining and Information Retrieval under the guidance of Prof. Sowmya Kamath}
    {A Graph Based Technique to Find Common News Contents From Multiple News Sources}
    {Surathkal, India}
    {January. 2014 - April. 2014}
    {
      \begin{cvitems}
        \item {Designed and implemented a graph based algorithm to find the common news contents.}
        \item {The project involved extracting content from RSS feeds of the news channels, NLP techniques to extract keywords and implementation of Document Distance Algorithm to calculate the Similarity Index between news items.}
        \item {Involved skills on Natural Language Processing, XML parser and Java.}
        \item {The project code can be accessed \href{https://github.com/varun93/graph-similarity}{\textbf{here}}.}
      \end{cvitems}
    }
    \cventry
    {Researcher for Popularity of Social Multimedia under the guidance of Prof. G. Ram Mohana Reddy}
    {Undergraduate Research, Popularity of Social Multimedia}
    {Surathkal, India}
    {July. 2014 - May. 2015}
    {
      \begin{cvitems}
        \item {Researched on the factors that affect the popularity of a social multimedia.}
        \item{For images we considered low level features, image content and social cues as the predictor variables and image views as the target variable. Used Spearman's rank correlation for evaluating the model.}
        \item {Implemented Support Vector Regression to predict views of an image, using python scikit-learn ML library. The images were queried using Flickr API, image tagging was done with Clarifai API.}
        \item {Designed and implemented alogorithms for dynamic real time clustering of Tweets using Twitter Streaming API. The implementation was done with python.}
        \item {The project code can be accessed \href{https://github.com/varun93/social-multimedia-popularity}{\textbf{here}}.}
      \end{cvitems} 
    }
 \cventry
   {Researcher for Web Service Discovery under the guidance of Prof. Sowmya Kamath}
    {Web Service Discovery and Recommendation}
    {Surathkal, India}
    {July. 2014 - December. 2014}
    {
     \begin{cvitems}
     \item {Implemented DBSCAN algorithm for clustering similar WSDL Files.}
     \item  {Evaluated the quality of clustering with Silhouette Coefficient.}
     \item {Involved skills in Natural Language Processing techniques for text analysis, DISCO for semantic similarity and python for implementation of the DBSCAN algorithm.}
     \end{cvitems} 
    }
\end{cventries}
